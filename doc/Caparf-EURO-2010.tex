%%%%%%%%%%%%%%%%%%%%%%%%%%%%%%%%%%%%%%%%%%%%%%%%%%%%%%%%%%%%%%%%%%%%%%%%%%%%%%%%
%% Copyright (C) 2010 Denis Nazarov <denis.nsc@gmail.com>.
%%
%% This file is part of caparf (http://code.google.com/p/caparf/).
%%
%% caparf is free software: you can redistribute it and/or modify
%% it under the terms of the GNU General Public License as published by
%% the Free Software Foundation, either version 3 of the License, or
%% (at your option) any later version.
%%
%% caparf is distributed in the hope that it will be useful,
%% but WITHOUT ANY WARRANTY; without even the implied warranty of
%% MERCHANTABILITY or FITNESS FOR A PARTICULAR PURPOSE. See the
%% GNU General Public License for more details.
%%
%% You should have received a copy of the GNU General Public License
%% along with caparf. If not, see <http://www.gnu.org/licenses/>.
%%%%%%%%%%%%%%%%%%%%%%%%%%%%%%%%%%%%%%%%%%%%%%%%%%%%%%%%%%%%%%%%%%%%%%%%%%%%%%%%

\documentclass[unicode]{beamer}
\usepackage{cmap}
\usepackage[T2A]{fontenc}
\usepackage[cp1251]{inputenc}
\usepackage[english,russian]{babel}
\usepackage{color}
\usetheme{Darmstadt}
\setbeamertemplate{footline}[page number]
\setbeamertemplate{navigation symbols}{} 

\title[Cutting And Packing Algorithms Research Framework]{
Cutting And Packing Algorithms\\
Research Framework\\
http://caparf.googlecode.com}
\author{Denis Nazarov}
\institute{Ufa State Technical University of Aviation}
\date{12 July 2010}

\begin{document}

\frame{\titlepage}

\section{Motivation}
\subsection{}

\begin{frame}{Algorithm development challenges/issues}
\Large
\begin{itemize}
	\item Algorithm details are often omitted in articles \bigskip
  \item It's hard to get/generate test instances proposed by other authors \bigskip
  \item Different system configurations for testing \bigskip
  \item Small number of review articles with algorithms comparisons \bigskip
\end{itemize}
\end{frame}

\begin{frame}{Desired solution}
\begin{block}{Consists of}
\begin{itemize}
	\item Algorithms (i.e. lower bounds/upper bounds/exact methods) for solving particular type(s) of problems
	\item Test data generators
\end{itemize}
\end{block}
\begin{block}{Provides}
\begin{itemize}
  \item Convenient �testing scenario� model
	\item Various reports generators (for example, results for article in TeX or presentation)
	\item Easy way to develop your own algorithms using existing methods and compare them with well-known reference algorithms by other authors
\end{itemize}
\end{block}
\end{frame}

\section{Implementation}
\subsection{}

\begin{frame}{Cutting and Packing Algorithms Research Framework}
\Large
\begin{enumerate}
	\item CAPARF is an OpenSource project\\ {\normalsize \itshape hosted at  \textcolor{blue}{http://code.google.com }under GPLv3 licence}
	\item CAPARF is a cross-platform framework\\ {\normalsize \itshape it is implemented in Java}
	\item CAPARF forces unification\\ {\normalsize \itshape all algorithms for the same type of problem implement the same interface}
	\item CAPARF potentially can contain a lot of algorithms and test
data/generators
\end{enumerate}
\end{frame}

\begin{frame}{Problem type concept}
Problem type is defined by {\ttfamily Input} and {\ttfamily Output}.
\begin{block}{{\ttfamily Input} is represented by}
\begin{itemize}
  \item ordered list of items to pack
  \item and, possibly, ordered list of containers into which items are packed
\end{itemize}
\end{block}
\begin{block}{{\ttfamily Output} is represented by}
\begin{itemize}
  \item ordered list of item placements
  \item objective function
\end{itemize}
\end{block}
{\ttfamily OutputVerifier} is implemented for each problem type, i.e. {\ttfamily Input} and {\ttfamily Output}.
\end{frame}

\begin{frame}{Algorithm concept}
Algorithm can be defined for particular problem type or for some class of problem types. \\\medskip
\begin{block}{Algorithm types}
\begin{enumerate}
	\item Lower bounds
	\item Upper bounds
	\item Exact methods
\end{enumerate}
\end{block} \medskip
Interruption of algorithm computations can be safely done by implementing {\ttfamily Interruptible} interface.
\end{frame}

\begin{frame}{Test data generator and scenario concepts}
Test data generator provides reference and random {\ttfamily Input}s.\\
Scenario is defined for particular problem type.
\begin{block}{Scenario components}
\begin{itemize}
	\item Algorithms of the same type to run
	\item Test data on which algorithms should be run
	\item Technical constraints (like time limit)
\end{itemize}
\end{block}
Report generator(s) can be added to produce statistics/results in preferable format.
\end{frame}

\section{Contribution}
\subsection{}

\begin{frame}{Contribution}
CAPARF can be successful and useful only as a community project.\\
\begin{block}{Possible types of contribution}
\begin{enumerate}
	\item Ideas, suggestions, criticism
	\item Feature requests, bug reports
	\item Code changes, including but not limited to: 
\begin{enumerate}
	\item Adding your algorithms, test generators
	\item Changing CAPARF core, adding more functionality
\end{enumerate}	
\end{enumerate}
\end{block}
\end{frame}

\begin{frame}{Contacts}
\Large For more information visit\medskip
\huge
\begin{center}
\textcolor{blue}{http://caparf.googlecode.com}\\
\end{center}

\bigskip\bigskip

\Large or mail to\medskip
\huge
\begin{center}
\textcolor{blue}{caparf-discuss@googlegroups.com}
\end{center}
\end{frame}

\end{document}